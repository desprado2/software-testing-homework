\documentclass[12pt, a4paper, oneside]{ctexart}
\usepackage{amsmath, amsthm, amssymb, appendix, bm, graphicx, hyperref, mathrsfs}

\CTEXsetup[format={\Large\bfseries}]{section}

\title{\textbf{测试计划}}
\author{第25组}
\date{\today}

\begin{document}


\maketitle
\section{测试需求}
本次作业要求使用黑盒测试,使用边界值、等价类、决策表的方法设计测试用例,根据结果定位软件缺陷。

我们小组选择的测试对象是在线进制转换工具(\url{https://www.sojson.com/hexconvert.html}),
该webapp有七个功能,分别是二进制转化、四进制转换、八进制转换、十进制转换、十六进制转换、三十二进制转换、六十四进制转换。
我们决定对除十进制转换外的六个功能进行测试。

\section{任务分配}
本小组为四人小组,任务分配为:谭博仁负责测试十六进制转换、结果分析和报告撰写,周峰负责测试二进制转换和四进制转换和八进制转换,
张怡天负责测试三十二进制转换和六十四进制转换。


\section{对十六进制转换功能的测试}

\subsection{测试策略分析}

在该功能中,输入为一个独立的、位于[0,0x4000000000000000]的十六进制整数。因此,采用边界值分析和等价类测试是较好的方法。

\subsection{测试用例}

\subsubsection{边界值分析}

边界值分析采用健壮性测试,即考虑无效值的情况。测试用例设计如下(下面所有输入输出均为正则表达式):
\begin{table}[!h]
    \small
    \begin{tabular}{|l|l|l|l|l|l|l|}
    \hline
    用例编号 & 输入 & 二进制 & 四进制 & 八进制 & 十进制 & 十六进制\\ \hline
    1 & -1 & UB & UB & UB & UB & UB \\ \hline
    2 & 0 & 0 & 0 & 0 & 0 & 0\\ \hline
    3 & 1 & 1 & 1 & 1 & 1 & 1 \\ \hline
    4 & 233 & 1000110011 & 20303 & 1063 & 563 & 233 \\ \hline
    5 & 3f\{15\} & 1\{62\} & 3\{31\} & 37\{20\} & 4611686018427388063 & 3f\{15\}\\ \hline
    6 & 40\{15\} & 10\{62\} & 10\{31\} & 40\{20\} & 4611686018427388064 & 40\{15\}\\ \hline
    7 & 40\{14\}1 & UB & UB & UB & UB & UB \\ \hline
    8 & 1.1 & UB & UB & UB & UB & UB\\ \hline
    9 & fg & UB & UB & UB & UB & UB\\ \hline
    \end{tabular}
\end{table}
其中UB表示undefined behaviour,出现UB时输出可以为任意值,但是整个网页不能出现崩溃、卡死等情况。

\newpage
上述用例的说明如下:
\begin{table}[!h]
    \begin{tabular}{|l|l|}
    \hline
    用例编号 & 说明\\ \hline
    1 & 输入略低于最小值\\ \hline
    2 & 输入等于最小值\\ \hline
    3 & 输入略高于最小值 \\ \hline
    4 & 输入为正常值 \\ \hline   
    5 & 输入略低于最大值\\ \hline
    6 & 输入等于最大值 \\ \hline
    7 & 输入略高于最大值 \\ \hline
    8 & 输入为小数 \\ \hline
    9 & 输入不是十六进制 \\ \hline
    \end{tabular}
\end{table}

\subsubsection{等价类分析}

参考我们一般情况下做进制转换的算法,可以将输入按照模16的余数(也就是输入的最后一位)进行分类,因为\textbf{进制转换时对每一位进行的计算过程应当是一致的}。据此可以设计如下测试用例:

\begin{table}[!h]
    \small
    \begin{tabular}{|l|l|l|l|l|l|l|}
    \hline
    用例编号 & 输入 & 二进制 & 四进制 & 八进制 & 十进制 & 十六进制\\ \hline
    10 & 40123456789abcdef & UB & UB & UB & UB & UB \\ \hline
    11 & -5 & UB & UB & UB & UB & UB \\ \hline
    12 & 510 & 10100010000 & 110100 & 2420 & 1296 & 510 \\ \hline
    13 & f1 & 11110001 & 3301 & 361 & 241 & f1 \\ \hline
    14 & a2 & 10100010 & 2202 & 242 & 162 & a2 \\ \hline
    15 & 903 & 100100000011 & 210003 & 4403 & 2307 & 903 \\ \hline
    16 & 5b4 & 10110110100 & 112310 & 2664 & 1460 & 5b4 \\ \hline
    17 & 2a5 & 1010100101 & 22211 & 1245 & 677 & 2a5  \\ \hline
    18 & 446 & 10001000110 & 101012 & 2106 & 1094 & 446  \\ \hline
    19 & 877 & 100001110111 & 201313 & 4167 & 2167 & 877  \\ \hline
    20 & 658 & 11001011000 & 121120 & 3130 & 1624 & 658  \\ \hline
    21 & aa9 & 101010101001 & 222221 & 5251 & 2729 & aa9  \\ \hline
    22 & 7da & 11111011010 & 133122 & 3732 & 2010 & 7da  \\ \hline
    23 & 4fb & 10011111011 & 103323 & 2373 & 1275 & 4fb  \\ \hline
    24 & 9fc & 100111111100 & 213330 & 4774 & 2556 & 9fc  \\ \hline
    25 & a3d & 101000111101 & 220331 & 5075 & 2621 & a3d  \\ \hline
    26 & 41e & 10000011110 & 100132 & 2036 & 1054 & 41e  \\ \hline
    27 & 6af & 11010101111 & 122233 & 3257 & 1711 & 6af  \\ \hline
    \end{tabular}
\end{table}
其中,用例10和11分别对应于高于最大值和低于最小值的输入。用例12~17对应于按照模16的余数划分的16个不同等价类的输入。

\section{对三十二进制转换功能的测试}

\subsection{测试用例}

\subsubsection{边界值分析}

边界值分析采用健壮性测试,即考虑无效值的情况。测试用例设计如下(下面所有输入输出均为正则表达式):
\begin{table}[!h]
    \small
    \begin{tabular}{|l|l|l|l|l|l|l|}
    \hline
    用例编号 & 输入 & 二进制 & 四进制 & 八进制 & 十进制 & 三十二进制\\ \hline
    1 & -1 & UB & UB & UB & UB & UB \\ \hline
    2 & 0 & 0 & 0 & 0 & 0 & 0\\ \hline
    3 & 1 & 1 & 1 & 1 & 1 & 1 \\ \hline
    4 & 114 & 10000100100 & 100210 & 2044 & 1060 & 114 \\ \hline
    5 & 3Z\{12\} & 10\{62\} & 10\{31\} & 40\{20\} & 4611686018427388064 & 3Z\{12\}\\ \hline
    6 & 40\{12\} & 10\{62\} & 10\{31\} & 40\{20\} & 4611686018427388064 & 40\{12\}\\ \hline
    7 & 40\{11\}1 & UB & UB & UB & UB & UB \\ \hline
    8 & 1.1 & UB & UB & UB & UB & UB\\ \hline
    9 & c++ & UB & UB & UB & UB & UB\\ \hline
    \end{tabular}
\end{table}
其中UB表示undefined behaviour,出现UB时输出可以为任意值,但是整个网页不能出现崩溃、卡死等情况。


上述用例的说明如下:
\begin{table}[!h]
    \begin{tabular}{|l|l|}
    \hline
    用例编号 & 说明\\ \hline
    1 & 输入略低于最小值\\ \hline
    2 & 输入等于最小值\\ \hline
    3 & 输入略高于最小值 \\ \hline
    4 & 输入为正常值 \\ \hline   
    5 & 输入略低于最大值\\ \hline
    6 & 输入等于最大值 \\ \hline
    7 & 输入略高于最大值 \\ \hline
    8 & 输入为小数 \\ \hline
    9 & 输入不是三十二进制 \\ \hline
    \end{tabular}
\end{table}

\subsubsection{等价类分析}

针对常见有符号数的不同长度(1 byte, 2 bytes, 4 bytes, 8 bytes)所对应的最大和最小值进行了测试,因为在有效范围内,一定大小范围内的输入应当有相同的处理模式。据此可以设计如下测试用例:

\begin{table}[!h]
    \small
    \begin{tabular}{|l|l|l|l|l|l|l|}
    \hline
    用例编号 & 输入 & 二进制 & 四进制 & 八进制 & 十进制 & 三十二进制\\ \hline
    10 & 1 & 1 & 1 & 1 & 1 & 1 \\ \hline
    11 & 3Z & 1111111 & 1333 & 177 & 127 & 3Z \\ \hline
    12 & 40 & 10000000 & 2000 & 200 & 128 & 80 \\ \hline
    13 & 1ZZZ & 1\{15\} & 33333333 & 177777 & 65535 & 1ZZZ \\ \hline
    14 & 2000 & 10\{15\} & 100000000 & 200000 & 65536 & 2000 \\ \hline
    15 & 1ZZZZZZ & 1\{31\} & 13\{15\} & 17\{10\} & 2147483647 & 1ZZZZZZ \\ \hline
    16 & 2000000 & 10\{31\} & 20\{15\} & 20\{10\} & 2147483648 & 2000000 \\ \hline
    17 & 3Z\{12\} & 10\{62\} & 10\{31\} & 40\{20\} & 4611686018427388064 & 3Z\{12\} \\ \hline
    \end{tabular}
\end{table}
选取了每一类的最大最小值作为用例进行测试。

其中,用例10和11分别对应于大小在$1\sim2^{8}-1$的输入,用例12和13分别对应于大小在$2^{8}\sim2^{16}-1$的输入,用例14和15分别对应于大小在$2^{16}\sim2^{32}-1$的输入,用例16和17分别对应于大小在$2^{32}\sim2^{62}-1$的输入。

\section{对六十四进制转换功能的测试}

\subsection{测试用例}

\subsubsection{边界值分析}

边界值分析采用健壮性测试,即考虑无效值的情况。测试用例设计如下(下面所有输入输出均为正则表达式):
\begin{table}[!h]
    \small
    \begin{tabular}{|l|l|l|l|l|l|l|}
    \hline
    用例编号 & 输入 & 二进制 & 四进制 & 八进制 & 十进制 & 六十四进制\\ \hline
    1 & -B & UB & UB & UB & UB & UB \\ \hline
    2 & A & 0 & 0 & 0 & 0 & 0\\ \hline
    3 & B & 1 & 1 & 1 & 1 & B \\ \hline
    4 & C++ & 10111110111110 & 2332332 & 27676 & 12222 & C++ \\ \hline
    5 & D/\{10\} & 10\{62\} & 10\{31\} & 40\{20\} & 4611686018427388064 & D/\{10\}\\ \hline
    6 & EA\{10\} & 10\{62\} & 10\{31\} & 40\{20\} & 4611686018427388064 & EA\{10\}\\ \hline
    7 & EA\{9\}B & UB & UB & UB & UB & UB \\ \hline
    8 & 1.1 & UB & UB & UB & UB & UB\\ \hline
    9 & -$\backslash$ & UB & UB & UB & UB & UB\\ \hline
    \end{tabular}
\end{table}
其中UB表示undefined behaviour,出现UB时输出可以为任意值,但是整个网页不能出现崩溃、卡死等情况。


上述用例的说明如下:
\begin{table}[!h]
    \begin{tabular}{|l|l|}
    \hline
    用例编号 & 说明\\ \hline
    1 & 输入略低于最小值\\ \hline
    2 & 输入等于最小值\\ \hline
    3 & 输入略高于最小值 \\ \hline
    4 & 输入为正常值 \\ \hline   
    5 & 输入略低于最大值\\ \hline
    6 & 输入等于最大值 \\ \hline
    7 & 输入略高于最大值 \\ \hline
    8 & 输入为小数 \\ \hline
    9 & 输入不是六十四进制 \\ \hline
    \end{tabular}
\end{table}

\subsubsection{等价类分析}

针对常见有符号数的不同长度(1 byte, 2 bytes, 4 bytes, 8 bytes)所对应的最大和最小值进行了测试,因为在有效范围内,一定大小范围内的输入应当有相同的处理模式。据此可以设计如下测试用例:

\begin{table}[!h]
    \small
    \begin{tabular}{|l|l|l|l|l|l|l|}
    \hline
    用例编号 & 输入 & 二进制 & 四进制 & 八进制 & 十进制 & 六十四进制\\ \hline
    10 & B & 1 & 1 & 1 & 1 & B \\ \hline
    11 & B/ & 1111111 & 1333 & 177 & 127 & B/ \\ \hline
    12 & CA & 10000000 & 2000 & 200 & 128 & CA \\ \hline
    13 & P// & 1\{15\} & 33333333 & 177777 & 65535 & P// \\ \hline
    14 & QAA & 10\{15\} & 100000000 & 200000 & 65536 & QAA \\ \hline
    15 & B///// & 1\{31\} & 13\{15\} & 17\{10\} & 2147483647 & B/////\\ \hline
    16 & CAAAAA & 10\{31\} & 20\{15\} & 20\{10\} & 2147483648 & CAAAAA \\ \hline
    17 & D/\{10\} & 10\{62\} & 10\{31\} & 40\{20\} & 4611686018427388064 & D/\{10\} \\ \hline
    \end{tabular}
\end{table}
选取了每一类的最大最小值作为用例进行测试。

其中,用例10和11分别对应于大小在$1\sim2^{8}-1$的输入,用例12和13分别对应于大小在$2^{8}\sim2^{16}-1$的输入,用例14和15分别对应于大小在$2^{16}\sim2^{32}-1$的输入,用例16和17分别对应于大小在$2^{32}\sim2^{62}-1$的输入。

\newpage
\section{对二进制转换功能的测试}

\subsection{测试用例}

\subsubsection{边界值分析}
边界值分析我们采用健壮性测试,考虑无效的勤快,测试用例如下(所有输入输出均为正则表达式)
\begin{table}[!h]
    \begin{tabular}{|l|l|l|l|l|l|l|}
    \hline
    用例编号 & 输入 & 二进制 & 四进制 & 八进制 & 十进制 & 十六进制 \\ \hline
    1 & -1 & UB & UB & UB & UB & UB \\ \hline
    2 & 0 & 0 & 0 & 0 & 0 & 0 \\ \hline
    3 & 1 & 1 & 1 & 1 & 1 & 1\\ \hline
    4 & 111 & 111 & 13 & 7 & 7 & 7 \\ \hline
    5 & 1\{52\}0 & 1\{52\}0 & 13\{25\}2 & 37\{16\}6 & 9007199254740990 & 1ffffffffffffe \\ \hline
    6 & 1\{53\} & 1\{53\} & 13\{26\} & 37\{17\} & 9007199254740991 & 1fffffffffffff \\ \hline
    7 & 10\{53\} & 10\{53\} & 20\{26\} & 40\{17\} & 9007199254740992 & 20000000000000 \\ \hline
    8 & 0.1 & UB & UB & UB & UB & UB \\ \hline
    9 & 34 & UB & UB & UB & UB & UB\\ \hline
    \end{tabular}
\end{table}
其中UB表示undefined behaviour,出现UB时输出可以为任意值,但是整个网页不能出现崩溃、卡死等情况。

\begin{table}[!h]
    \begin{tabular}{|l|l|}
    \hline
    用例编号 & 用例说明\\ \hline
    1 & 输入略低于最小值\\ \hline
    2 & 输入等于最小值 \\ \hline
    3 & 输入略高于最小值 \\ \hline
    4 & 输入为正常值 \\ \hline
    5 & 输入略低于最大值 \\ \hline
    6 & 输入等于最大值 \\ \hline
    7 & 输入略高于最大值 \\ \hline
        
    \end{tabular}
\end{table}

\newpage

\subsubsection{等价类分析}

由于整个进制转换上限是53位,按照位数划分,每相邻四位为一个等价类,最后53位单独为一个等价类,然后与几个类位数相同的无效输入。

\begin{table}[!h]
    \begin{tabular}{|l|l|l|l|l|l|l|}
    \hline
    用例编号 & 输入 & 二进制 & 四进制 & 八进制 & 十进制 & 十六进制 \\ \hline
    10 & 11 & 11 & 3 & 3 & 3 & 3 \\ \hline
    11 & 1\{5\} & 1\{5\} & 133 & 37 & 31 & 1f \\ \hline
    12 & 1\{9\} & 1\{9\} & 13333 & 777 & 511 & 1ff \\ \hline
    13 & 1\{13\} & 1\{13\} & 1333333 & 17777 & 8191 & 1fff \\ \hline
    14 & 1\{17\} & 1\{17\}& 13\{8\} & 37777 & 131071 & 1ffff \\ \hline
    15 & 1\{21\}& 1\{21\} & 13\{10\} & 7\{7\}& 2097151 & 1f\{5\} \\ \hline
    16 & 1\{25\} & 1\{25\}& 13\{12\} & 17\{8\} & 33554431 & 1f\{6\}\\ \hline
    17 & 1\{29\} & 1\{29\} & 13\{14\} & 37\{9\} & 536870911 &1f\{17\} \\ \hline
    18 & 1\{33\}& 1\{33\} & 13\{16\} &7\{11\} &8389914591 & 1f\{8\}\\ \hline
    19 & 1\{37\}& 1\{37\}&13\{18\} & 17\{12\} & 137438953471 & 1f\{9\}\\ \hline
    20 & 1\{41\}& 1\{41\} &13\{20\} &7\{14\} &  36321726838276 &1f\{10\} \\ \hline
    21 & 1\{45\}& 1\{45\} &13\{22\} & 17\{15\}& 562948853728311 &1f\{11\} \\ \hline
    22 & 1\{49\}& 1\{49\}&13\{24\} &37\{16\} &  9001839243740991 & 1f\{12\}\\ \hline
    23 & 1\{53\}& 1\{53\} &13\{26\} &17\{17\} &210183324923641521 &1f\{13\} \\ \hline
    24 & 1\{54\}& 10\{53\} & 10\{27\} &10\{13\} & 18014398509381827 &40\{14\} \\ \hline
    25 & 3\{10\}&UB &UB &UB &UB & UB\\ \hline
    \end{tabular}
\end{table}
    选取了每一类中的具有代表性的作为用例进行测试。例10对应$0\sim2^{4}-1$的输入,例11对应$2^{4}\sim2^{8}-1$的输入,
    例12对应$2^{8}\sim2^{12}-1$的输入,例13对应$2^{12}\sim2^{16}-1$的输入,例14对应$2^{16}\sim2^{20}-1$的输入,
    例15对应$2^{20}\sim2^{24}-1$的输入,例16对应$2^{24}\sim2^{28}-1$的输入,例17对应$2^{28}\sim2^{32}-1$的输入,
    例18对应$2^{32}\sim2^{36}-1$的输入,例19对应$2^{36}\sim2^{40}-1$的输入,例20对应$2^{40}\sim2^{44}-1$的输入,
    例21对应$2^{44}\sim2^{48}-1$的输入,例22对应$2^{48}\sim2^{52}-1$的输入,例23对应$2^{52}\sim2^{53}-1$的输入,
    例24对应$2^{53}\sim$的输入,例25对应无规则的输入.

\newpage
\begin{table}[!h]
    \begin{tabular}{|l|l|}
    \hline
    用例编号 & 用例说明\\ \hline
    10 & 1~4\\ \hline
    11 & 5~8 \\ \hline
    12 & 9~12 \\ \hline
    13 & 13~16 \\ \hline
    14 & 17~20 \\ \hline
    15 & 21~24 \\ \hline
    16 & 25~28 \\ \hline
    17 & 29~32  \\ \hline
    18 & 33~36 \\ \hline
    19 & 37~40 \\ \hline
    20 & 41~44 \\ \hline
    21 & 45~48 \\ \hline
    22 & 49~52 \\ \hline
    23 & 53 \\ \hline
    24 & 范围上限 \\ \hline
    25 & 输入不符合要求 \\ \hline

    \end{tabular}
\end{table}





\section{对四进制转换功能的测试}

\subsection{用例分析}

\subsubsection{边界值分析}
边界值分析我们采用健壮性测试,考虑无效的勤快,测试用例如下(所有输入输出均为正则表达式)

\begin{table}
\begin{tabular}{|l|l|l|l|l|l|l|}
\hline
用例编号 & 输入 & 二进制 & 四进制 & 八进制 & 十进制 & 十六进制\\ \hline
1 & -1 & UB & UB & UB & UB & UB \\ \hline
2 & 0 & 0 & 0 & 0 & 0 & 0 \\ \hline
3 & 1 & 1 & 1 & 1 & 1 & 1 \\ \hline
4 & 111 & 10101 & 111 & 25 & 21 & 15 \\ \hline
5 & 13\{25\}2 & 1\{52\}0 & 13\{25\}2 & 37\{16\}6 & 9007199436876719 & 1f\{12\}e \\ \hline
6 & 13\{26\} & 1\{53\} & 13\{26\} & 37\{17\}\ &    9007199234740991 & 1f\{13\} \\ \hline
7 & 20\{26\} & 10\{53\} & 10\{26\} & 20\{17\} &    4503599627370496 & 10\{14\}  \\ \hline
8 & 0.1 & UB & UB & UB & UB & UB  \\ \hline
9 & fd & UB & UB & UB & UB & UB  \\ \hline

\end{tabular}
\end{table}
其中UB表示undefined behaviour,出现UB时输出可以为任意值,但是整个网页不能出现崩溃、卡死等情况。
\begin{table}[!h]
    \begin{tabular}{|l|l|}
    \hline
    用例编号 & 说明\\ \hline
    1 & 输入略低于最小值\\ \hline
    2 & 输入等于最小值\\ \hline
    3 & 输入略高于最小值 \\ \hline
    4 & 输入为正常值 \\ \hline   
    5 & 输入略低于最大值\\ \hline
    6 & 输入等于最大值 \\ \hline
    7 & 输入略高于最大值 \\ \hline
    8 & 输入为小数 \\ \hline
    9 & 输入不是四进制 \\ \hline
    \end{tabular}
\end{table}


\newpage

\subsection{等价类划分}

\begin{table}[!h]
    \begin{tabular}{|l|l|l|l|l|l|l|}
    \hline
    用例编号 & 输入 & 二进制 & 四进制 & 八进制 & 十进制 & 十六进制 \\ \hline
    10 & 3 & 11 & 3 & 3 & 3 & 3 \\ \hline
    11 & 3\{5\} & 1\{10\} & 33333 & 1777 & 1023 & 3ff \\ \hline
    12 & 3\{9\} & 1\{18\} & 3\{9\} & 777777 & 262143 & 3ffff \\ \hline
    13 & 3\{13\} & 1\{26\} & 3\{13\} & 37777777 & 67108856 & 3ffffff \\ \hline
    14 & 3\{17\} & 1\{34\}& 3\{17\} & 377777777 & 17179869183 & 3fffffff \\ \hline
    15 & 3\{21\}& 1\{42\} & 3\{21\} & 7\{11\}& 4398672611186 & 3f\{10\} \\ \hline
    16 & 13\{26\} & 1\{52\}& 3\{26\} & 17\{16\} & 4572567782638127 & f\{13\}\\ \hline
    17 & 0.1 & UB &UB &UB &UB &UB \\ \hline
    18 & fd &UB &UB &UB &UB &UB \\ \hline
    \end{tabular}
\end{table}
选取了每一类中的具有代表性的作为用例进行测试。例10对应$0\sim2^{8}-1$的输入,例11对应$2^{8}\sim2^{16}-1$的输入,
例12对应$2^{16}\sim2^{24}-1$的输入,例13对应$2^{24}\sim2^{32}-1$的输入,例14对应$2^{32}\sim2^{40}-1$的输入,
例15对应$2^{40}\sim2^{48}-1$的输入,例16对应$2^{48}\sim2^{53}-1$的输入,例17代表小数点输入,
例18对应无规则输入。



\section{对八进制转换功能的测试}
\subsection{测试用例}
\subsubsection{边界值分析}
边界值分析我们采用健壮性测试,考虑无效的勤快,测试用例如下(所有输入输出均为正则表达式)

\begin{table}

\begin{tabular}{|l|l|l|l|l|l|}
\hline
用例编号 & 输入 & 二进制 & 四进制  & 十进制 & 十六进制\\ \hline
1 & -1 & UB & UB & UB  & UB \\ \hline
2 & 0 & 0 & 0 & 0  & 0 \\ \hline
3 & 1 & 1 & 1 &  1 & 1 \\ \hline
4 & 666 & 110110110 & 12312 & 436 & 1b6 \\ \hline
5 & 37\{16\}6 & 1\{52\}0  & 13\{25\}2 & 9007199243730990 & 1f\{12\}e \\ \hline
6 & 37\{17\}  & 1\{53\} & 13\{26\}\}\ &    9007199243730991 & 1f\{13\} \\ \hline
7 & 40\{17\} & 10\{26\} & 20\{17\} &    90073218797134974 & 20\{13\}  \\ \hline
8 & 0.1 & UB & UB  & UB & UB  \\ \hline
9 & fd & UB & UB  & UB & UB  \\ \hline

\end{tabular}
\end{table}
其中UB表示undefined behaviour,出现UB时输出可以为任意值,但是整个网页不能出现崩溃、卡死等情况。
\begin{table}[!h]
    \begin{tabular}{|l|l|}
    \hline
    用例编号 & 说明\\ \hline
    1 & 输入略低于最小值\\ \hline
    2 & 输入等于最小值\\ \hline
    3 & 输入略高于最小值 \\ \hline
    4 & 输入为正常值 \\ \hline   
    5 & 输入略低于最大值\\ \hline
    6 & 输入等于最大值 \\ \hline
    7 & 输入略高于最大值 \\ \hline
    8 & 输入为小数 \\ \hline
    9 & 输入不是八进制 \\ \hline
    \end{tabular}
\end{table}


\subsubsection{等价类划分}

\newpage
\begin{table}[!h]
    \begin{tabular}{|l|l|l|l|l|l|}
    \hline
    用例编号 & 输入 & 二进制 & 四进制  & 十进制 & 十六进制 \\ \hline
    10 & 7 & 111 & 13 & 7 &  7 \\ \hline
    11 & 7\{5\} & 1\{15\} & 13\{7\} & 32767 & 7fff \\ \hline
    12 & 7\{9\} &  1\{27\} & 13\{13\} & 134217727 & 7f\{6\} \\ \hline
    13 & 7\{13\} & 1\{39\} &  13\{19\} & 549822743887 & 7f\{9\} \\ \hline
    14 & 7\{17\} & 1\{51\}&  13\{25\} & 2251677641676347 & 7f\{12\} \\ \hline
    15 & 37\{17\}& 1\{53\} &  13\{26\}& 144115188978622772 & 1f\{13\} \\ \hline
    16 & 0.1 & UB &UB &UB  &UB \\ \hline
    17 & c++ &UB &UB &UB &UB \\ \hline
    \end{tabular}
\end{table}
选取了每一类中的具有代表性的作为用例进行测试。例10对应$0\sim2^{12}-1$的输入,例11对应$2^{12}\sim2^{24}-1$的输入,
例12对应$2^{24}\sim2^{32}-1$的输入,例13对应$2^{32}\sim2^{48}-1$的输入,例14对应$2^{48}\sim2^{53}-1$的输入,
例15对应$2^{53}\sim∞$的输入,例16对应小数的输入,例17代表无规则输入,
例18对应无规则输入。

\end{document}
