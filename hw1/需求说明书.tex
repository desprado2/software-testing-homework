\documentclass[12pt, a4paper, oneside]{ctexart}
\usepackage{amsmath, amsthm, amssymb, appendix, bm, graphicx, hyperref, mathrsfs}

\CTEXsetup[format={\Large\bfseries}]{section}

\title{\textbf{软件需求规格说明书}}
\author{第25组}
\date{\today}

\begin{document}


\maketitle
\section{概述}
“进制转换工具”是一款webapp,提供在线的各种进制转换功能。由于该软件并非我们小组成员开发,也无现有的需求说明书,因此我们根据网页上的描述,以及
翻看部分代码,根据自己理解定义了该软件需求规格说明书。

\section{功能性需求}
该软件支持二进制转化、四进制转换、八进制转换、十进制转换、十六进制转换、三十二进制转换、六十四进制转换七个独立模块,分别提供不同进制的转换功能。

\section{可用性需求和可靠性需求}
每个功能模块中,在输入框中输入一个对应进制的数字,可以在下面的输出框中输出该数字对应的七个不同进制的表示。
例如在二进制转换功能中输入“100”,输出包含“二进制:100,四进制:10,八进制:4,十进制:4,十六进制:4,三十二进制:4,六十四进制:E”。

当输入不满足软件适用范围限制时,输出可以为任意值,但是网站不出现崩溃、卡死等情况。

\section{性能}
输入数字后必须在0.1秒内输出结果。

\section{适用范围限制}
由于网站上未写明软件的适用范围,我们翻看网站代码后,结合实际情况,定义软件的适用范围如下:

在二进制转换、四进制转换、八进制转换中,输入的数字要求为整数,且对应的十进制在$[0,2^{53}-1]$范围内。

在十进制转换、十六进制转换、三十二进制转换、六十四进制转换中,输入的数字要求为整数,且对应的十进制在$[0,2^{62}-1]$范围内。

\end{document}